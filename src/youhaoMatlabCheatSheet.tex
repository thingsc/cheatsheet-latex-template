\documentclass[8pt]{innovativeinnovation-cheatsheet}
\definecolor{myred}{RGB}{250,127,111}
\usepackage{enumitem}
\usepackage{graphicx}
\graphicspath{{../img/}}
\newcommand{\myinline}[1]{{\color{innoinnored}\bfseries\ttfamily{#1}}}
\newtheorem{assumption}{Assumption}
\newcount\myfigurecount
\newcount\mytablecount
\myfigurecount = 1
\mytablecount = 1

\lstset{
    basicstyle=\ttfamily,
    breaklines=true,
    breakatwhitespace=true,
    tabsize=4,
    showstringspaces=true,
    extendedchars=true,
    inputencoding=utf8,
    frame=single,
    language=Matlab,
    captionpos=b,
    keywordstyle=\color{innoinnored}\bfseries,
%     numbers=left,
%     numberstyle=\tiny\color{gray},
}



\cheatsheettitle{MATLA Cheat Sheet --- Youhao HU}

\begin{document}

\begin{multicols*}{3}


\cheatsheetsection{MATLAB Command line}

\begin{enumerate}[label=$\bullet$,leftmargin=*,nosep]
    \item \lstinline!dir!: list files and folders in current folder
    \item \lstinline!dir('*.txt')!: find txt files in current folder
    \item \lstinline!pwd!: show current folder
    \item \lstinline!cd!:  change current folder
    \item \lstinline!delete('filename')!: delete file
    \item {\color{innoinnored}\bfseries\ttfamily movefile}\lstinline{('oldpath/oldname','newpath/newname')}: rename or move file    
    \item \lstinline!what!: list Matlab/simulink files in current folder
    \item {\color{innoinnored}\bfseries\ttfamily more on}: display output one screen at a time, space to continue, q to quit
    \item {\color{innoinnored}\bfseries\ttfamily more off}: display output continuously
\end{enumerate}

\cheatsheetsection{Find files using Regular Expression}


\begin{enumerate}[label=$\bullet$,leftmargin=*,nosep]
    \item \myinline{dir}\verb|('*.txt')|: Find the files with extension ``txt''
    \item \myinline{dir}\verb|('*motor*')|: Find the files with ``motor'' in the file name
\end{enumerate}


\begin{center}
Table \the\mytablecount. Some rules of the regular expression
\advance\mytablecount by 1

\begin{tabular}{cp{.8\linewidth}}
\hline
Code                        & Meaning\\
\hline
\myinline{*}                & Match any characters\\
\myinline{.}                & Match any single character\\
\myinline{\textbackslash w} & Match a word character (letter/digit/underscore/Chinese character)\\
\myinline{\textbackslash s} & Match a single space\\
\myinline{\textbackslash d} & Match a single digit\\
\myinline{\textbackslash b} & Match a single word boundary (Namely the beginning/end of a word)\\
\myinline{$\hat{}$}         & Match the beginning of a line\\
\myinline{\$}               & Match the end of a line\\
\hline
\end{tabular}
\end{center}


\cheatsheetsection{Legend/title rendered in \LaTeX}

\begin{lstlisting}
h = legend(legend1,legend2);% for instance, legend1 = $x_1$
h.Interpreter = 'latex';
h.FontSize    = fontsize;
h.Location    = 'northeast';
h.Orientation = 'horizon';
title('nameoftitle','interpreter','latex')% for instance, name = $V_o$
\end{lstlisting}

\cheatsheetsection{Save figure}


\begin{lstlisting}
saveas(gcf,'filename','png'); % No resolution ratio option
print(gcf,'-dpng', '-r300', 'filename'); % 300 dpi png
\end{lstlisting}

\cheatsheetsection{Data interpolation / Prediction of missing data}

Match the data \myinline{iL1out} with the simulation time vector \myinline{toutSIM}.
\begin{enumerate}[label=$\bullet$,leftmargin=*,nosep]
    \item Original data: \myinline{(t,iL1out)}
    \item New data: \myinline{(toutSIM,iL1out)}
\end{enumerate}

\begin{lstlisting}
iL1out = interp1(t,iL1out,toutSIM)
\end{lstlisting}


\begin{enumerate}[label=$\bullet$,leftmargin=*,nosep]
    \item Original data: \myinline{(t,iL1out)}
    \item Wanted point: \myinline{(t1,iL1)}
\end{enumerate}

\begin{lstlisting}
iL1 = interp1(t,iL1out,t1)
\end{lstlisting}


\cheatsheetsection{The greatest common divisor / The least common multiple}

\begin{enumerate}[label=$\bullet$,leftmargin=*,nosep]
    \item \lstinline!gcd(a,b)!: The greatest common divisor of a and b 
    \item \lstinline!lcm(a,b)!: The least common multiple of a and b
\end{enumerate}

\cheatsheetsection{GA to tune parameters in Simulink}
\begin{lstlisting}
clc,clear,close all

SimulinkModel = 'DCMotorPID'; % Simulink model name, case-insensitive.
open(SimulinkModel);
fun = @GATestFun;
nvars = 4;
lb = [0;0;0;20000];
ub = [.1;.1;.1;40000];
MaxGen  = 1;
PopSize = 10; % Large size brings better result, but takes more time.
options = optimoptions('ga', 'PopulationSize', PopSize, 'MaxGenerations', MaxGen, 'Display', 'iter');
[x,fval,exitflag] = ga(fun, nvars, [], [], [], [], lb, ub, [], options);


%------------------------------------------
%--- Display the parameters tuned by GA ---
%------------------------------------------

kp = x(1)
ki = x(2)
kd = x(3)
fN = x(4)


%------------------------------------------
%--- Show the result ---
%------------------------------------------

hws = get_param(SimulinkModel,'modelworkspace');
hws.assignin('kp',kp);
hws.assignin('ki',ki);
hws.assignin('kd',kd);
hws.assignin('fN',fN);

simout = sim(SimulinkModel);

t          = simout.simout.Time;
SpeedError = simout.simout.Data;
plot(t,SpeedError,'LineWidth',1.5)
title('$Tracking Error$','interpreter','latex')
grid on

%-----------------------------------------------
%--- The cost function to evaluate the error ---
%-----------------------------------------------

function cost = GATestFun(inputpara)

SimulinkModel = 'DCMotorPID';
kp = inputpara(1);
ki = inputpara(2);
kd = inputpara(3);
fN = inputpara(4);

hws = get_param(SimulinkModel, 'modelworkspace');
hws.assignin('kp',kp)
hws.assignin('ki',ki)
hws.assignin('kd',kd)
hws.assignin('fN',fN)
simout = sim(SimulinkModel);
cost   = rms(simout);
end
\end{lstlisting}


\begin{center}
\includegraphics[width = \linewidth]{DCMotorSimulinkModel}
Figure 1. Simulink Model
\end{center}

\begin{center}
Table 1. Parameters in the Simulink Model
\begin{tabular}{cc||cc}
\hline
Parameters & Values & Parameters & Values\\
\hline
\myinline{Ke} & 0.1                     & \myinline{fai} & 0.1\\
\myinline{Kt} & \myinline{Ke} * 30 / pi & \myinline{J}   & 0.001\\
\myinline{L}  & 0.005                   & \myinline{B}   & 0.01 \\
\myinline{R}  & 0.1                     & \myinline{TL}  & 0\\
\hline
\end{tabular}
\end{center}


\cheatsheetsection{Resample the data}

Suppose there is a data set $(t,V_{RL})$ obtained from SIMULINK, we want to resample it to reduce the data size:

\begin{lstlisting}
told   = simuout.out.simout.Time;
VRLold = simuout.out.simout.Data(:,1);

dtold  = mean(diff(t)); % Original sampling time
dtnew  = .2e-3;         % New even sampling time
Q      = round(dtnew/dtold); % Resample rate
tnew   = t(1):dtnew:length(VRL);
VRLnew = resample(simuout.out.simout.Data(:,1),1,Q);

\end{lstlisting}

\cheatsheetsection{Solve function}


\begin{lstlisting}
syms x
eq = x^2 == 1;
solve(eq,x)
\end{lstlisting}


\cheatsheetsection{Solve function numerically}


\begin{lstlisting}
syms x
eq = sin(x)^2 == 1;
vpasolve(eq,x,5) % 5 digits, default is 32 digitss
\end{lstlisting}

\cheatsheetsection{Symbol value to numemrical value}


\begin{lstlisting}
syms x
solSym = solve(x^2 == 1,x)
solNum = vpa(solSym,5) % 5 digits, default is 32 digits
\end{lstlisting}

\cheatsheetsection{Subsitute value into symbolic expression}


\begin{lstlisting}
syms x
subs(x^2,x,2)
\end{lstlisting}

\cheatsheetsection{Symbolic to double}


\begin{lstlisting}
syms x
SymVar = int(x^2,x,0,1);
VpaVar = vpa(SymVar,5); % 5 digits, default is 32 digits
DblVar = double(VpaVar);
\end{lstlisting}

\cheatsheetsection{Polynomial fit}


\begin{lstlisting}
x = 0:0.1:.5;
y = sin(x);
p = polyfit(x,y,1); % 1st order polynomial fit

plot(x,y)
hold on
plot(x,p(1) * x + p(2))
\end{lstlisting}

\cheatsheetsection{Troubleshooting}

Q: SIMULINK does not work properly even with identical parameters.

A: Check the Model workspace and the Base workspace. The Model workspace has higher priority than the Base workspace. If the same variable is defined in both workspaces, the value in the Model workspace will be used.







\vfill

\cheatsheetfooter{Innovative Innovation}{https://github.com/innovativeinnovation}

\end{multicols*}

\end{document}
