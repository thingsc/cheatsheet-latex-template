\documentclass[8pt]{innovativeinnovation-cheatsheet}
\definecolor{myred}{RGB}{173, 0, 21}
\usepackage{enumitem}
\usepackage{amsmath}
\usepackage{circuitikz}
\usepackage{siunitx}
\tikzset{>=latex} % for LaTeX arrow head	
\ctikzset{resistors/scale=0.5}			
\ctikzset{capacitors/scale=0.5}
\ctikzset{inductors/scale=0.5}
\ctikzset{sources/scale=0.6}
\ctikzset{diodes/scale=0.6}

\def\killdepth#1{{\raisebox{0pt}[\height][0pt]{#1}}}
\def\coord(#1){coordinate(#1)} 
\def\coord(#1){coordinate(#1) node[circle, red, draw, inner sep=1pt, pin={[red, overlay, inner sep=0.5pt, font=\tiny, pin distance=0.1cm, pin edge={red, overlay,-}]45:#1}](#1-node){}}	
% Uncomment the above line to show the coordinate node in the circuit

\newcount\myfigurecount
\newcount\mytablecount
\myfigurecount = 1
\mytablecount  = 1

\cheatsheettitle{CircuiTikz Gallery --- Youhao HU}

\begin{document}

\begin{multicols*}{3}

\cheatsheetsection{SS-Type Wireless Power Transfer}

\begin{center}
\centering
\begin{tikzpicture}[american, line width=0.2mm]
\def\w{3}
\def\h{2}
\def\wT{1}

\draw (0,0) 
to[sV, l_=$V_1$] ++(0,\h) \coord(U1top)
to[C, l_=$C_1$, i=$I_1$] ++(\w,0) 
to[L, l_=$L_1$, name=L1] ++(0,-\h) \coord(end1)
to[R, l_=$R_1$] (0,0);	

\draw (end1)++(\wT,0) \coord(begin2)
to[L, l_=$L_2$, name=L2] ++(0,\h) 
to[C, l_=$C_2$, i<=$I_2$] ++(\w,0)
to[R, l_=$R_\text{L}$] ++(0,-\h)
to[R, l_=$R_2$] (begin2);

\draw
[<->,violet!90] (L1.core west) \coord(L1cw) to[out=60, in=120]
node[midway, above]{$M$} (L2.core east) \coord(L2ce);	
\end{tikzpicture}
\end{center}


\cheatsheetsection{Inverter}

\begin{center}
    \begin{circuitikz}[american,line width = .2mm]
        \draw (0,0) \coord(origin);
        \draw (0,2.7) \coord(U1);
        \draw (U1) to[V, l = $V_{DC}$] (origin);
        \draw (U1) to[short,-*] ++(1.5,0) \coord(U2);
        \draw (U2) -- ++(1.5,0) \coord(U3);
        \draw (U2) --++(0,-.2) \coord(M1);
        \draw (M1) node[nigfete, bodydiode, fill = black, scale = .6, anchor = D](S1){S1};
        \draw (S1.S) to[short, -*]++(0,-.25) \coord(M21);
        \draw (S1.S) --++(0,-.5) \coord(M2);
        \draw (M2) node[nigfete, bodydiode, fill = black, scale = .6, anchor = D](S2){S2};
        \draw (S2.S) --++(0,-.1) \coord(M3);
    
    
        \draw (U3) --++(0,-.2) \coord(SM1);
        \draw (SM1) node[nigfete, bodydiode, fill = black, scale = .6, anchor = D](S4){S4};
        \draw (S4.S) to[short, -*]++(0,-.25) \coord(SM21);
        \draw (S4.S) -- ++(0,-.5) \coord(SM2);
        \draw (SM2) node[nigfete, bodydiode,  fill = black, scale = .6, anchor = D](S3){S3};
        \draw (S3.S) --++(0,-.1) \coord(SM3);
        \draw (origin) to[short,-*]  (M3|-origin)\coord(B1);
        \draw (B1) -- (SM3|-origin) \coord(B2);
    
        \draw(M3) -- (B1);
        \draw(SM3) -- (B2);
    
        \draw(M21) to[R] (SM21);
        % \draw[help lines, red, step = .5cm] (origin) grid ++(U3);
    \end{circuitikz}
\end{center}

\cheatsheetsection{Full Bridge Rectifier}

\begin{center}
\begin{circuitikz}[american, line width = .2mm]
\draw (0,0)\coord(origin);
% \draw [help lines, red, step=.5cm] (0,-.5) grid (4,2);
\draw (0,2) \coord (U1);
\draw (2,2) \coord(U2);
\draw (2,0) \coord(B1);
\draw (origin) to [sV] (U1);
\draw (U1) to[short, -*] (U2);
\path (U2) --++(-1,-1) \coord(C1);
\path (U2) --++(1,-1) \coord(C2);
\draw (C1) to[empty diode, fill = myred, color = myred] (U2);
\draw (U2) to [full diode] (C2);
\draw (C1) to [empty diode] (B1);
\draw (B1) to [empty diode] (C2);
\draw (origin) to[short,-*] (B1);
\draw (C2) to[short,-*] (C2);

\draw (C2) --++(1,0) \coord(C3);
\path (C3) --++(0,-1.5) \coord(B2);
\draw (C3) to[R] (B2);
\draw (B2) -- (C1|-B2) \coord(B3);
\path (B3) -- (B3|-origin) \coord(B4);
\draw (B4) node[jump crossing, rotate = 90] (J1){};
\draw (J1.west) -- (B3);
\draw (J1.east) to[short,-*] (C1);
\end{circuitikz}
\end{center}


\cheatsheetsection{Buck converter with DC machine}

\begin{center}
\begin{circuitikz}[american,line width = .2mm]


\ctikzset{voltage dir=RP}
\ctikzset{batteries/scale = .8}
\ctikzset{transistors/scale = .8}
\ctikzset{inductors/scale = .6}
\ctikzset{capacitors/scale = .6}
\ctikzset{diodes/scale = .6}
\ctikzset{electromechanicals/scale = .8}


\draw (0,0) \coord(origin);
\path (origin) ++ (4,2) \coord(U1);
% \draw[help lines, red, step = 1cm] (origin) grid ++(U1); % help grid
\draw (origin) to[battery1] (U1-|origin) \coord(U2);
\draw (U2) to[short] ++(1,0) \coord(U3);
\draw (U3) node[nigfete, bodydiode, fill = black, anchor=D, rotate = 90](S1){};
\draw (S1.S) -- ++(1,0) \coord(U4);
\draw (S1.G) node[left] {$D$};
\draw (U4) to[L, l_ = $L_b$] ++ (1,0)\coord(U5);
\draw (U5) to[short, -*] ++(1,0)\coord(U6);
\draw (U6) to[C, l_ = $C_b$] (U6|-origin)\coord(B1);
\draw (B1) to[short, -*] ++(-2.7,0) \coord(B2);
\draw (B2) to[diode, fill = black, l_ = $D_b$] (U2-|B2) \coord(U7);
\draw (U7) to[short, -*] (U7);
\draw (B2) -- (origin);
\draw (B1) -- (U1|-B1)\coord(B3);
\draw (B1) --++(1,0) \coord(B4);
\draw (B1) to[short, -*] (B1);
\draw (U6) -- (U6-|B4) \coord(U8);
\draw (U8) --++ (0,-1) \coord(M1);
\draw (M1) node[elmech](motor){M};
\draw (motor.south) -- (B4);
\end{circuitikz}
\end{center}





\vfill

\cheatsheetfooter{Innovative Innovation}{https://github.com/innovativeinnovation}

\end{multicols*}

\end{document}
