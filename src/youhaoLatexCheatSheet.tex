\documentclass[8pt]{innovativeinnovation-cheatsheet}
\definecolor{myred}{RGB}{250,127,111}
\usepackage{enumitem}

\usepackage[os=mac]{menukeys}
\renewmenumacro{\keys}[+]{shadowedroundedkeys}
\renewmenumacro{\menu}[>]{roundedmenus}
\renewmenumacro{\path}[/]{pathswithfolder}

\newcount\myfigurecount
\newcount\mytablecount
\myfigurecount = 1
\mytablecount  = 1

\newcommand{\myinline}[1]{{\color{innoinnored}\bfseries\ttfamily{#1}}}
\newtheorem{assumption}{Assumption}
\lstset{
    basicstyle=\ttfamily,
    breaklines=true,
    breakatwhitespace=true,
    tabsize=4,
    showstringspaces=true,
    extendedchars=true,
    inputencoding=utf8,
    frame=single,
    language=[LaTeX]TeX,
    captionpos=b,
    keywordstyle=\color{innoinnored}\bfseries,
%     numbers=left,
%     numberstyle=\tiny\color{gray},
}



\cheatsheettitle{\LaTeX \, Cheat Sheet --- Youhao HU}

\begin{document}

\begin{multicols*}{3}

\cheatsheetsection{Two columns}

\begin{lstlisting}
\begin{columns}
\begin{column}{.5\linewidth}
Here is the content in column 1
\end{column}
\begin{column}{.5\linewidth}
Here is the content in column 2
\end{column}
\end{columns}
\end{lstlisting}

\cheatsheetsection{Footnote in two columns}

Footnote mark is in one of the column, but the footnote is at the bottom of the page.

\begin{lstlisting}
\begin{frame}
\begin{columns}
\begin{column}{.5\linewidth}
Here is the content in column 1 \footnotemark[1]
\end{column}
\begin{column}{.5\linewidth}
Here is the content in column 2
\end{column}
\end{columns}
\footnotetext[1]{This is the first footnote}
\end{frame}
\end{lstlisting}



\cheatsheetsection{Figure with 1 title}

\begin{lstlisting}
\begin{figure}
\caption{HKUSTGZ Logo}
\includegrapics[width=.5\linewidth](USTGZ.png)
\label{fig:USTGZLogo}
\end{figure}
\end{lstlisting}

\cheatsheetsection{Figure path setting}

Set the images path as \verb|./images/|

\begin{lstlisting}
\usepackage{graphicx}
\graphicspath{{images/}}
\end{lstlisting}

\columnbreak

\cheatsheetsection{Figure options}
\begin{lstlisting}
\includegraphics[scale=0.1](USTGZ.png)
\includegraphics[width=.5\linewidth](USTGZ.png)
\includegraphics[height=1cm](USTGZ.png)
\end{lstlisting}

\cheatsheetsection{Figure with 2+ titles (no label)}
\begin{lstlisting}
\begin{center}
\includegraphics[width=.4\linewidth]{SMC.png}
\includegraphics[width=.4\linewidth]{TSMC.png}\\
SMC (Left) and TSMC (Right) comparison
\end{center}
\end{lstlisting}


\cheatsheetsection{Figure with 2+ titles (1 label)}

\begin{lstlisting}
\begin{figure}
\centering
\includegraphics[width=.4\linewidth]{SMC.png}
\includegraphics[width=.4\linewidth]{TSMC.png}\\
\caption{SMC (Left) and TSMC (Right) comparison}
\label{fig:TSMCom}
\end{figure}
\end{lstlisting}

\cheatsheetsection{Figure with 2+ titles (2+ labels)}


\begin{lstlisting}
\usepackage{graphicx}
\usepackage{subcation}

\begin{figure}
\caption{SMC with changing load}
\begin{subfigure}{.4\linewidth}
\caption{Scope}
\includegraphics[width=\linewidth] {SMC_15_25_Scope.png}
\end{subfigure}
\begin{subfigure}{.4\linewidth}
\caption{MATLAB}
\includegraphics[width=\linewidth] {SMC_15_25.png}
\end{subfigure}
\label{fig:Tracking}
\end{figure}
\end{lstlisting}



\cheatsheetsection{Remark environment}
\begin{lstlisting}
\newtheorem{assumption}{Assumption}
\newtheorem{remark}{Remark}
\newtheorem{figurebox}{\quad} % To render a box only, used in beamer
\end{lstlisting}
Which is rendered as:
\begin{assumption}
    This is a new assumption.
\end{assumption}

\cheatsheetsection{Break the column}
\begin{enumerate}[label=$\bullet$,leftmargin=*,nosep]
    \item \myinline{\textbackslash columnbreak}
\end{enumerate}

\cheatsheetsection{DIY your own counter}

\begin{lstlisting}
\newcount\myfigurecount
\newcount\mytablecount
\myfigurecount = 1
\mytablecount = 1
\end{lstlisting}

Usage:

\begin{lstlisting}
Table \the\mytablecount. Some rules of the regular expression
\advance\mytablecount by 1
Table \the\mytablecount. Other rules of the regular expression
\end{lstlisting}

Rendered as below:

Table 1. Some rules of the regular expression

Table 2. Other rules of the regular expression

\cheatsheetsection{VSCODE usilization}


\begin{center}
    Table \the\mytablecount. Some shortcut keys in VSCODE
    \advance\mytablecount by 1

\begin{tabular}{cl}
    \hline
    Keys & Description\\
    \hline
\keys{\cmd + \shift + p} & Open command palette\\
\keys{\cmd + \shift + x} & Open extension tab\\
\keys{\cmd + \Alt + v}   & Open internal PDF viewer\\
\keys{\cmd + \Alt + z}   & Line-wrap switch\\
\keys{\cmd + \Alt + x}   & Open LaTeX Workshop plate\\
\keys{\cmd + clicking}   & Navigate content from PDF to tex file\\

\hline
\end{tabular}
\end{center}

\cheatsheetsection{Fancy keys/manus/paths}



\begin{lstlisting}
\usepackage{menukeys}
\renewmenumacro{\keys}[+]{shadowedroundedkeys}
\renewmenumacro{\menu}[>]{roundedmenus}
\renewmenumacro{\path}[/]{pathswithfolder}

\menu{Extras > Settings > {Units, rulers and origin}}
\path{/Users/youhao/mdfiles/research\_related/ latex\_cheatsheet.tex}
\keys{\Alt + \shift + x}

\end{lstlisting}

Which is rendered as:

\menu{Extras > Settings > {Units, rulers and origin}}

\path{/Users/youhao/mdfiles/research\_related/latex\_cheatsheet.tex}

\keys{\Alt + \shift + x}


\cheatsheetsection{CircuitTikz vertical line}


\begin{lstlisting}
\begin{circuitikz}
\draw (0,0) \coord(origin);
\draw (2,2) \coord(U1);
\draw (U1) -- (origin-|U1) \coord(B1);
\draw (origin) -- (B1);
\end{circuitikz}
\end{lstlisting}

\vfill

\cheatsheetfooter{Innovative Innovation}{https://github.com/innovativeinnovation}

\end{multicols*}

\end{document}
