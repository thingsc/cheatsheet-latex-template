\documentclass[8pt]{innovativeinnovation-cheatsheet}
\definecolor{myred}{RGB}{250,127,111}
\usepackage{enumitem}
\usepackage{amsmath}
\newcommand{\myinline}[1]{{\color{innoinnored}\bfseries\ttfamily{#1}}}

\usepackage[os=mac]{menukeys}
\renewmenumacro{\keys}[+]{shadowedroundedkeys}
\renewmenumacro{\menu}[>]{angularmenus}

\lstset{
    basicstyle=\ttfamily,
    breaklines=true,
    breakatwhitespace=true,
    tabsize=4,
    showstringspaces=true,
    extendedchars=true,
    inputencoding=utf8,
    frame=single,
    language=Mathematica,
    captionpos=b,
    keywordstyle=\color{innoinnored}\bfseries,
%     numbers=left,
%     numberstyle=\tiny\color{gray},
}



\cheatsheettitle{Mathematica Cheat Sheet --- Youhao HU}

\begin{document}

\begin{multicols*}{3}


\cheatsheetsection{Mathematica Command line}

\begin{enumerate}[leftmargin=*,nosep]
    \item[$\bullet$] \lstinline!Clear!\myinline{["`*"]}: Clear all the variables.
    \item[$\bullet$] \myinline{Clear[w0]}: Clear the variable w0.
    \item[$\bullet$] \myinline{w0=.}: Clear the variable w0.
    \item[$\bullet$] \myinline{\%}: The last calculation result.
    \item[$\bullet$] \keys{\esc + a + \esc}: Insert $\alpha$.
    \item[$\bullet$] \myinline{\textbackslash[Alpha]}: Insert $\alpha$.
\end{enumerate}

\cheatsheetsection{Matrix and array}
\begin{equation}
    \begin{gathered}
        A = \left[\begin{array}{ccc}
            1 & 2 & 3 \\
            4 & 5 & 6 \\
            7 & 8 & 9
        \end{array}\right],\quad
        B = \left[\begin{array}{ccc}
            1 & 2 & 3 \\
            4 & 5 & 6 \\
            7 & 8 & 9
        \end{array}\right]\\
        C= AB
    \end{gathered}
\end{equation}
Rendered as the below code in Mathematica:
\begin{lstlisting}
A = {{1, 2, 3}, {4, 5, 6}, {7, 8, 9}};
B = {{1, 2, 3}, {4, 5, 6}, {7, 8, 9}};
C = A.B;
\end{lstlisting}

\cheatsheetsection{Plot}   

Plot the function $V = \frac{4}{\pi}\frac{V_{in}}{Z_{in2}}$, $Z_{in2} =   \omega M + \frac{R_1 (R_2 + R_e)}{\omega M}
$ from $Re=0\Omega$ to $Re=10\Omega$ (Assume that all the other variables are defined except $R_e$):
\begin{lstlisting}
Zin2 = w M + (R1 * (R2 + Req)) / w / M;
V2   = 4 / Pi Vin / Zin2;
Plot[V2, {Req,0,10}]
\end{lstlisting}

\cheatsheetsection{Plot first-order differential equation}
\begin{equation}
    \dot x + 3x = 0,\quad x(0) = 1
\end{equation}
Plot the solution $x(t)$ from $t=0$ to $t=10$.
\begin{lstlisting}
Clear[x]
xout = NDSolveValue[{x'[t]+3*x[t]==0,x[0]==1}, x, {t,0,10}]
Plot[xout[t],{t,0,10},PlotRange->All]
\end{lstlisting}

\cheatsheetsection{Plot second-order differential equation}
\begin{equation}
    \ddot y + \sin(y) = 0,\quad y(0) = 2,\quad \dot y(0) = 1
\end{equation}

Plot the solution $y(t)$ from $t=-10$ to $t=10$:

\begin{lstlisting}
Clear[y]
ode1 = {y''[t] + Sin[y[t]] == 0,y[0]==2,y'[0]==1};
sol = NDSolve[ode1,y,{t,-10,10}]
Plot[y[t] /. sol, {t,-10,10}] (* For only y *)
Plot[Evaluate[{y[t],y'[t],y''[t]} /. sol], {t,-10,10}] (* For y, ydot and yddot*)
\end{lstlisting}


Dolor sed viverra ipsum nunc aliquet bibendum. Senectus et netus et malesuada
fames ac turpis egestas. Pellentesque id nibh tortor id aliquet.

\vfill

\cheatsheetfooter{Innovative Innovation}{https://github.com/innovativeinnovation}

\end{multicols*}

\end{document}
